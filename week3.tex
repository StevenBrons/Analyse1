\section{Week 3}

\stelling{Bolzano-Weierstrass: Stel $(a_n)_{n=0}^\infty$ is een begrensde rij van re\'ele getallen. Dan is er een deelrij $(a_{n_k})^\infty_{k=1}$ die convergeert.}
\bewijs{
	$(a_n)_{n=0}^\infty$ is begrensd, \\
	dus er zijn $c_0 < d_0 \in \R$ zdd $a_n\in [c_0,d_0] \forall n$\\
	def $e_i = \frac{c_i+d_i}{2}$\\
	def $I_i = \{a_n|n\in\N\wedge a_n\in[c_i,e_i]\}$\\
	def $J_i = \{a_n|n\in\N\wedge a_n\in[e_i,d_i]\}$\\
	Nu ittereren we $e_i,I_i,J_i$ door nieuwe $c_i,d_i$ te bepalen:\\
	\als{
		$I_i$ oneindig is:\\
			$c_{i+1}=c_i$\\
			$d_{i+1}=e_i$\\
	}
	\als{
		$I_i$ eindig is: (dan is $J_i$ oneindig)\\
			$c_{i+1}=e_i$\\
			$d_{i+1}=d_i$\\
	}
	Dit geeft dus $(c_n)_{n\in\N},(d_n)_{n\in\N}$ zdd:\\
	$c_n < d_n$\\
	$d_n - c_n = \frac{d_0-c_0}{2^n}$
	waarbij $\frac{d_0-c_0}{2^n} \xrightarrow{n \to \infty} 0$\\
	dus $c_n=d_n$ als $n\to\infty$\\
	$(c_n)$ is monotoon stijgend\\
	$(d_n)$ is monotoon dalend\\
	$c_n \geq d_0 \wedge d_n \leq c_0$\\
	Dus $c_n \to l \wedge d_n \to m$\\
	Maar zoals we zagen $l = m$\\
	Definieer nu $n_1$ zdd $a_{n_1}\in[c_1,d_1]$ en \\
	$n_k$ zdd $a_{n_k}\in[c_k,d_k] \wedge n_k > n_{k-1}$\\
	Dus we kunnen $n_1<n_2<n_3<...$ kiezen zdd $a_{n_k}\in[c_k,d_k]$\\
	Ofwel: $c_k \leq a_{n_k} \leq d_k$, dus samen met $c_k\to l \wedge d_k\to l$\\
	volgt uit het sandwich principe: $a_{n_k}\to l$
}

\section*{Week 3}
\stelling{
	$(a_n)_{n=0}^\infty$ en $(b_n)_{n=0}^\infty$ zijn rijen reële getallen, dan geldt:
	\begin{enumerate}[label=\roman*)]
	\item{
		Als $\forall_na_n=l$ Dan $a_n\to l$ als $n\to\infty$
	}
	\item{
		Als $(a_n)_{n=0}^\infty$ een nulrij en $(b_n)_{n=0}^\infty$ begrensd dan is $(a_nb_n)_{n=0}^\infty$ een nulrij
	}
	\item{
		Als $a_n\to a$ en $b_n\to b$ als $n\to\infty$ dan $(a_n+b_n)\to a+b$ als $n\to\infty$
	}
	\item{
		Als $a_n\to a$ dan $c\cdot a_n\to c\cdot a$
	}
	\item{
		Als $a_n\to a$ en $b_n\to b$ dan $a_nb_n\to ab$
	}
	\item{
		Als $a_n\to a;a_n\neq0;a\neq0$ dan $1/a_n\to1/a$
	}
	\item{
		Als $a_n\to a;b_n\to b;\forall_na_n\leq b_n$ dan $a\leq b$
	}
	\item{
		Als $a_n\to a$ dan $a_{n_k}\to a$ voor alle deelrijen $a_{n_k}$
	}
	\end{enumerate}.
}
\bewijs{@Steven: enumerate wordt niet geaccepteerd binnen\\bewijzen,zelf geen oplossing kunnen vinden}
\begin{enumerate}[label=\roman*)]
	\item{
		Voor alle $\epsilon>0$ neem $n_0=0$
	}
	\item{
		$\exists_M\forall_n\left|b_n\right|<M$\\
		$\forall_{\epsilon>0}\exists_{n_0}\forall_{n\geq n_0}\left|a_n\right|<\epsilon/M$\\
		$\forall_{\epsilon>0}\exists_{n_0}\forall_{n\geq n_0}\left|a_nb_n\right|<\epsilon$
	}
	\item{
		Voor een gegeven $\epsilon>0$ geldt:\\
		$\exists_{n_0}\forall_{n\geq n_0}|a_n-a|<\epsilon/2$ en $\exists_{m_0}\forall_{m\geq m_0}|b_m-b|<\epsilon/2$\\
		Dus dan geldt $\forall_{n\geq\max(n_0,m_0)}|a_n+b_n-a-b|$\\$\leq|a_n-a|+|b_n-b|\leq\epsilon$
	}
	\item{
		Er zijn 2 gevallen: $c=0$ is triviaal en $c\neq0$:
		Gegeven $\epsilon>0$ geldt $\exists_{n_0}\forall{n\geq n_0}|a_n-a|<\epsilon/|c|$
		Dus $\forall_{\epsilon>0}\exists_{n_0}\forall_{n\geq n_0}|ca_n-ca|<\epsilon$
	}
	\item{
		$a_nb_n-ab=(a_n-a)(b_n-b)+(a_n-a)b+a(b_n-b)$\\
		$(a_n-a)\to0;(b_n-b)\text{ begrensd}\Rightarrow(a_n-a)(b_n-b)\to0$\\
		$(a_n-a)\to0\implies(a_n-a)b\to0$\\
		$(b_n-b)\to0\implies a(b_n-b)\to0$\\
		$a_nb_n-ab\to0$
	}
	\item{
		$|1/a_na|$ is begrensd dus $|a_n-a|/|a_na\to0$ en dus $|1/a_n-1/a|\to0$
	}
	\item{
		Stel $a>b$ dan $\exists_{n_0}\forall_{n\geq n_0}|(a_n-b_n)-(a-b)|<a-b$ maar dan geldt $a_n-b_n>0$ \lightning\\
		$a\leq b$
	}
	\item{
		Gegeven $\epsilon>0$ geldt $\exists_{N}\forall_{n\geq N}|a_n-a|<\epsilon$ en er geldt $\exists_{k_0}\forall_{k\geq k_0}n_k>N$
		Dus $\forall_{\epsilon>0}\exists_{k_0}\forall_{k\geq k_0}|a_{n_k}-a|<\epsilon$
	}
\end{enumerate}
