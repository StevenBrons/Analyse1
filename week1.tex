\section{Week 1}

\definitie{$\leq$ is totale ordening}{
	\als{$\leq$ partiele ordening}
	\uitwerking{
		$\forall x: x \leq x$ (reflexief)\\
		$\forall x,y: x \leq y \wedge y \leq z \Rightarrow x \leq z$ (transitief)\\
		$\forall x,y: x \leq y \wedge y \leq x \Rightarrow x = y$ (antisymetrish)
	}
	\als{$\leq$ totaal is}
	\uitwerking {
		$\forall x,y: x \leq y \wedge y \leq x$
	}
}
\definitie{$F$ is geordend lichaam} {
Een lichaam $(F,+,*,0,1)$ met een totale ordening $\leq$ op $F$\\
\als{$\forall a,b,c \in F: a \leq b \Rightarrow a + c \leq b + c$}
\als{$\forall a,b,c \in F: a \leq b \wedge c \geq 0 \Rightarrow ac \leq bc$}
}
\definitie{$b$ is bovengrens} {
$A \subseteq X, b \in X$\\
\als{$\forall x \in X: a \leq b$}
}
\definitie{$A$ is naar boven begrensd} {
\als{A heeft bovengrens}
}
\definitie{$a$ is grootste element} {
$A \subseteq X, a \in A$\\
\als{$a$ is bovengrens}
}
\definitie{supremum/kleinste bovengrens A}{
Zij $A \subseteq X, b \in X$\\
\als{$b$ bovengrens $A$}
\als{$c \in X$ bovengrens $A, \forall c:b \leq c$}
}
\definitie{volledig totaal geordende verzameling $(X,\leq)$} {
	\als{Elke naar boven begrensde deelverzameling $A \subseteq X$ een supremum heeft}
}
\feit{$(\mathbb{Q},\leq)$ is niet volledig}
\feit{$\mathbb{R}$ is de completering van $(\mathbb{Q}$}
\definitie{S is een Dedekind snede}{
	\als{$S \subseteq \mathbb{Q}$}
	\als{$\mathbb{Q} \neq S \neq \emptyset$}
	\als{$x,y \in \mathbb{Q}: x\leq y \wedge y \in S \Rightarrow x \in s$}
	\als{$S$ heeft geen grootste element}
	\uitwerking{
		$\forall x\in S  \exists y \in S $ zdd. $y > x$
	}
}
\feit{$S,T \in \mathbb{R}: S\leq T \Leftrightarrow S \subseteq T$}
\feit{$(\mathbb{R},+,*,\leq)$ is een volledig geordend lichaam}
\stelling{Als $R_1,R_2$ volledige geordende lichamen zijn, dan is er een bijectie $\alpha: R_1 \to R_2$ die alle structuren behoudt}
\uitwerking{
	$x,y \in R_1$\\
	\als{$\alpha(x + y) = \alpha(x) + \alpha(y)$}
	\als{$\alpha(xy) = \alpha(x)\alpha(y)$}
	\als{$a,b \in R_2: x \leq y \Leftrightarrow a \leq b$}
}

