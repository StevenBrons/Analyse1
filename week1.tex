\section{Week 1}
\begin{multicols}{2}

\definitie{$\leq$ is totale ordening}{
	\als{$\leq$ partiele ordening}\\
	\uitwerking{
		$\forall x: x \leq x$ (reflexief)\\
		$\forall x,y: x \leq y \wedge y \leq z \Rightarrow x \leq z$ (transitief)\\
		$\forall x,y: x \leq y \wedge y \leq x \Rightarrow x = y$ (antisymetrish)
	}\\
	\als{$\leq$ totaal is}\\
	\uitwerking {
		$\forall x,y: x \leq y \vee y \leq x$
	}
}

\definitie{geordend lichaam} {
Een lichaam $(F,+,*,0,1)$ met een totale ordening $\leq$ op $F$\\
\als{$\forall a,b,c \in F: a \leq b \Rightarrow a + c \leq b + c$}\\
\als{$\forall a,b,c \in F: a \leq b \wedge c \geq 0 \Rightarrow ac \leq bc$}
}

\definitie{$b$ is bovengrens} {
$A \subseteq X, b \in X$\\
\als{$\forall x \in X: a \leq b$}
}

\definitie{$A$ is naar boven begrensd} {
\als{A heeft bovengrens}
}

\definitie{$a$ is grootste element} {
$A \subseteq X, a \in A$\\
\als{$a$ is bovengrens}
}

\definitie{supremum/kleinste bovengrens A}{
Zij $A \subseteq X, b \in X$\\
\als{$b$ bovengrens $A$}\\
\als{$c \in X$ bovengrens $A, \forall c:b \leq c$}
}

\definitie{volledig totaal geordende verzameling $(X,\leq)$} {
	\als{Elke naar boven begrensde deelverzameling \\ $A \subseteq X$ een supremum heeft}
}

\feit{$(\Q,\leq)$ is niet volledig}

\feit{$\R$ is de completering van $(\Q$}

\definitie{S is een Dedekind snede}{
	\als{$S \subseteq \Q$}\\
	\als{$\Q \neq S \neq \emptyset$}\\
	\als{$x,y \in \Q: x\leq y \wedge y \in S \Rightarrow x \in s$}\\
	\als{$S$ heeft geen grootste element}\\
	\uitwerking{
		$\forall x\in S  \exists y \in S $ zdd. $y > x$
	}
}

\feit{$S,T \in \R: S\leq T \Leftrightarrow S \subseteq T$}

\feit{$(\R,+,*,\leq)$ is een volledig geordend lichaam}

\feit{Zij F geordend lichaam, dan $\forall x \ in F: x^2 \geq 0$ en $1 \geq 0$}
\bewijs{
	\als{$x \geq 0$ dan $x^2=x*x\geq 0$}\\
	\als{$x < 0$ dan $x^2=(-x)^2 > 0$}\\
	\als{$1 = 1^2 \geq 0 \wedge 1 \neq 0 \Rightarrow 1 > 0$}
}

\stelling{Als $R_1,R_2$ volledige geordende lichamen zijn, dan is er een bijectie $\alpha: R_1 \to R_2$ die alle structuren behoudt}
\uitwerking{
	$x,y \in R_1$\\
	\als{$\alpha(x + y) = \alpha(x) + \alpha(y)$}\\
	\als{$\alpha(xy) = \alpha(x)\alpha(y)$}\\
	\als{$a,b \in R_2: x \leq y \Leftrightarrow a \leq b$}
}\\

\lemma{Zij $F$ een willekeurig lichaam, dan is er een unieke afbeelding $\alpha: \Q \to F$ zdd $\alpha(0) = 0, \alpha(1) = 1, \alpha(m + n) = \alpha(m) + \alpha(m)$}
\bewijs{
	$\alpha(0)=0, \alpha(1)=1 \alpha(s(x)) = \alpha(x) + 1$\\*
	Deze afbeelding voldoet automatisch aan: \\*$\alpha(x + y) = \alpha(x) + \alpha(y)$
}

\definitie{Als $\alpha: \N_0 \to F$ injectief is dan heeft $F$ karakteristiek nul}{
	Dus: $\N_0 \subseteq F$
}

\feit{Elk geordend lichaam heeft karakteristiek nul}
\bewijs{
$\alpha(0)=0, \alpha(1)=1 \alpha(s(x)) = \alpha(x) + 1$\\
$n \in \N_0: \alpha(n + 1) = \alpha(s(n)) = \alpha(n) + 1 > \alpha(n)$\\
$\alpha(0) < \alpha(1) < \alpha(2) ...$\\
Dus $n \neq m \Leftrightarrow \alpha(n) \neq \alpha(m)$ dus injectie
}

\stelling{$\alpha: \N_0 \to F$ kunnen we uitbreiden naar $\alpha: \Z \to F$}
\uitwerking{
	$(m,n)\in \Z$\\
	Neem $\alpha(n,m)=\alpha(n)-\alpha(m)$\\
	Als $(m,n)\sim (a,b) \Rightarrow \alpha((m,n)) = \alpha((a,b))$
}\\

\stelling{$\alpha: \Z \to F$ kunnen we uitbreiden naar $\alpha: \Q \to F$}
\uitwerking{
	$(m,n) \to \frac{\alpha(m)}{\alpha(n)}$\\
	Dit kan alleen als $\alpha(m) \neq 0$\\
	Dit kan alleen als $F$ karakteristiek nul heeft\\
	Dus als $F$ een geordend lichaam is $\Rightarrow F$\\ heeft karakteristiek nul $\alpha$ is injectie.
}\\

\definitie{Een geordend lichaam heet Archimedis}{
	\als{$\forall x \in F  \exists n \in \N$ zdd $x \leq \alpha(n)$}
}

\stelling{Elk volledig geordend lichaam is Archimedis}
\bewijs{
\als{$x \leq 0$ dan is $n = 0$ goed}\\
\als{$x > 0$}\\
\uitwerking{
	Neem $j = \{\frac{1}{n}|n\in\N\}$\\
	Duidelijk: $0$ is een ondergrens voor $j$ want\\ $\forall n \in \N: 0 \leq \frac{1}{n}$\\
	\lemma{$0$ is infimum voor $J$}
	\uitwerking{
		$J$ heeft ondergrens $l$ (want 0)\\
		$J$ heeft infimum (def. volledigheid:\\
		elke deelverzameling met een ondergrens\\
		heeft een kleinste ondergrens)\\
		$l = inf(J) \geq 0$ (want 0 is ondergrens)\\
		Stel $l > 0$\\
		dan $2l = l + l > l \Rightarrow 2l$ is geen ondergrens\\
		$\exists n \in \N$ zdd $\frac{1}{n} < 2l \Rightarrow J \ni \frac{1}{2n} < l$\\
		Tegenspraak dus: $l = 0$
	}\\
	\als{$x > 0$}\\
	\uitwerking{
	dan $\frac{1}{x} > 0$ dus $\frac{1}{x}$ is geen ondergrens voor\\
	$j$ (want 	$inf(J)=0$\\
	dus $\exists n\in \N$\\
	$\frac{1}{n} < \frac{1}{x}$ dus $n > x$
	}
}
}

\stelling{Zij $F$ volledig geordend lichaam: $x,y \in F \wedge x < y: \exists r \in \Q $ zdd $x < r < y$}
\bewijs{
	\als{$x < 0 < y: r = 0$}\\
	\als{$x < y \leq 0 \Rightarrow 0 \leq -y < -x$}\\
	\uitwerking{
		Stel de uitspraak is bewezen voor\\ $0 \leq x \leq y \Rightarrow -y < r < -x\Rightarrow x < -r < y$\\
		Stel $0 \leq x < y \Rightarrow \exists n \in \N$ zdd $\frac{1}{n} < y - x$.(archimedis)\\
		Def $A = \{k \in \N_0|k \leq nx\} \neq \emptyset$ (archimedis)\\
		Def $a = max(A) \in \N_0$: nu $a \leq nx < a + 1$\\
		$\frac{a}{n} \leq x < \frac{a+1}{n=r}$\\
		$r = \frac{a}{n} + \frac{1}{n} < x + (y-x) < y$
	}
}

\propositie{Zij $F$ volledig geordend lichaam: $x,y \in F \wedge x < y: \exists r \in \Q $ zdd $x < r < y$}
\bewijs{
	\als{$x < 0 < y: r = 0$}\\
	\als{$x < y \leq 0 \Rightarrow 0 \leq -y < -x$}\\
	\uitwerking{
		Stel de uitspraak is bewezen voor\\
		$0 \leq x \leq y \Rightarrow -y < r < -x\Rightarrow x < -r < y$\\
		Stel $0 \leq x < y \Rightarrow \exists n \in \N$ zdd $\frac{1}{n} < y - x$.(archimedis)\\
		Def $A = \{k \in \N_0|k \leq nx\} \neq \emptyset$ (archimedis)\\
		Def $a = max(A) \in \N_0$: nu $a \leq nx < a + 1$\\
		$\frac{a}{n} \leq x < \frac{a+1}{n=r}$\\
		$r = \frac{a}{n} + \frac{1}{n} < x + (y-x) < y$
	}
}

\end{multicols}