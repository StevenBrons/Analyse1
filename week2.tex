\section{Week 2}
\begin{multicols}{2}

\stelling{Zij K,L volledig geordende lichamen, dan is er een bijectie $\alpha:K \to L$ die de ordening behoudt en de lichaamsoperatie}
\uitwerking{
Er is maar \'e\'en geordend lichaam
$x,y \in K$\\
$x < y \Rightarrow \alpha(x) < \alpha(y)$
$\alpha(0) = 0$
$\alpha(x+y) = \alpha(x) + \alpha(y)$
}
\bewijs{
(Schets)
Zij $x\in K$ def $C_x = \{r\in\Q|r < x\}\subseteq \Q$\\
$K$ is Archimedis, dus $\exists n\in\N$ zdd $n \geq x$\\
Nu is $n$ is een bovengrens voor $C_x$\\
$(n\in \N) \in C_x \subseteq \Q \subseteq L$\\
$L$ is volledig, dus elke naar boven begrensde\\
verzameling heeft een supremum\\
$y_x=sup(C_x)\in L$\\
$x \in K \wedge y_x \in L$ dus $K \xrightarrow{\alpha} L$\\
$\alpha$ voldoet aan de eisen:\\
\als{$x < y$}\\
\uitwerking{
Moet gelden dat: $\Rightarrow \alpha(x) < \alpha(y)$\\
Zij $x,y\in K$ met $x < y$\\
Dan $\exists r \in \Q$ zdd $x < r < s < y$\\
Dus $r \in C_y$ maar $r \not\in C_x$\\
$C_x \subseteq C_y$, dus $C_x \subset C_y$\\
dus $C_x < C_y$ (orde van $\R$)\\
dus $\alpha(x) = sup(C_x)\leq r < s \leq sup(C_y) = \alpha(y)$\\
dus $\alpha(x)$ is injectief en orde behoudend!
}\\
Omgekeerd werkt dit ook met $\beta: L \to K$\\
is een inverse afbeelding van $\alpha$\\
dus $\beta(\alpha(x))=x$ en $\alpha(\beta(y))=y$\\
dus $\alpha \wedge \beta$ zijn bijecties!\\
$\beta(\alpha(x))=sup(\{r\in \Q |r < sup(\{s\in \Q | s<x\})\})=x$
}

\feit{$\frac{1}{n}$ is een nulrij}
\bewijs{
	gegeven een $\epsilon>0$ dan:\\
	$\epsilon>0\implies \frac{1}{\epsilon}>0\implies \exists_{n\in\mathbb{R}}n>\frac{1}{\epsilon}\implies\frac{1}{n}<\epsilon$.\\
	We weten dat deze $n$ bestaat omdat $\mathbb{R}$ archimedisch is.\\
	voor $N\geq n$: $N\geq n\implies\frac{1}{N}\leq\frac{1}{n}<\epsilon\implies\frac{1}{N}<\epsilon$
}

\feit{als een rij een limiet heeft dan is deze uniek}
\bewijs{
	stel $k,l$ zijn limieten: $\{a_n\}\to k \land \{a_n\}\to l$ dan:\\
	$\forall_{\epsilon>0}\exists_{n\in\mathbb{R}}\forall_{N>n}(\left|a_N-k\right|<\epsilon\land\left|a_N-l\right|<\epsilon)$\\
	neem dan $\epsilon=\frac{\left|k-l\right|}{3}$:\\
	$\left|a_N-k\right|<\frac{\left|k-l\right|}{3}\implies a_N<k+\frac{\left|k-l\right|}{3}$\\
	$\implies \frac{\left|k-l\right|}{3}>\left|a_N-l\right|>\frac{2\left|k-l\right|}{3}\implies k-l=0\implies k=l$\\
	Dus als er een limiet bestaat, is deze uniek
}


\end{multicols}