\section{Week 2}
\begin{multicols}{2}


\stelling{Zij K,L volledig geordende lichamen, dan is er een bijectie $\alpha:K \to L$ die de ordening behoudt en de lichaamsoperatie}
\uitwerking{
Er is maar \'e\'en geordend lichaam
$x,y \in K$\\
$x < y \Rightarrow \alpha(x) < \alpha(y)$
$\alpha(0) = 0$
$\alpha(x+y) = \alpha(x) + \alpha(y)$
}
\bewijs{
(Schets)
Zij $x\in K$ def $C_x = \{r\in\Q|r < x\}\subseteq \Q$\\
$K$ is Archimedis, dus $\exists n\in\N$ zdd $n \geq x$\\
Nu is $n$ is een bovengrens voor $C_x$\\
$(n\in \N) \in C_x \subseteq \Q \subseteq L$\\
$L$ is volledig, dus elke naar boven begrensde\\
verzameling heeft een supremum\\
$y_x=sup(C_x)\in L$\\
$x \in K \wedge y_x \in L$ dus $K \xrightarrow{\alpha} L$\\
$\alpha$ voldoet aan de eisen:\\
\als{$x < y$}\\
\uitwerking{
Moet gelden dat: $\Rightarrow \alpha(x) < \alpha(y)$\\
Zij $x,y\in K$ met $x < y$\\
Dan $\exists r \in \Q$ zdd $x < r < s < y$\\
Dus $r \in C_y$ maar $r \not\in C_x$\\
$C_x \subseteq C_y$, dus $C_x \subset C_y$\\
dus $C_x < C_y$ (orde van $\R$)\\
dus $\alpha(x) = sup(C_x)\leq r < s \leq sup(C_y) = \alpha(y)$\\
dus $\alpha(x)$ is injectief en orde behoudend!
}\\
Omgekeerd werkt dit ook met $\beta: L \to K$\\
is een inverse afbeelding van $\alpha$\\
dus $\beta(\alpha(x))=x$ en $\alpha(\beta(y))=y$\\
dus $\alpha \wedge \beta$ zijn bijecties!\\
$\beta(\alpha(x))=sup(\{r\in \Q |r < sup(\{s\in \Q | s<x\})\})=x$
}




\end{multicols}