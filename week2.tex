\section{Week 2}
\begin{multicols}{2}

\stelling{Zij K,L volledig geordende lichamen, dan is er een bijectie $\alpha:K \to L$ die de ordening  en de lichaamsoperatie behouden}
\uitwerking{
Er is maar \'e\'en geordend lichaam
$x,y \in K$\\
$x < y \Rightarrow \alpha(x) < \alpha(y)$,
$\alpha(0) = 0$,
$\alpha(x+y) = \alpha(x) + \alpha(y)$
}
\bewijs{
(Schets)
Zij $x\in K$ def $C_x = \{r\in\Q|r < x\}\subseteq \Q$\\
$K$ is Archimedis, dus $\exists n\in\N$ zdd $n \geq x$\\
Nu is $n$ is een bovengrens voor $C_x$\\
$(n\in \N) \in C_x \subseteq \Q \subseteq L$\\
$L$ is volledig, dus elke naar boven begrensde\\
verzameling heeft een supremum\\
$y_x=sup(C_x)\in L$\\
$x \in K \wedge y_x \in L$ dus $K \xrightarrow{\alpha} L$\\
$\alpha$ voldoet aan de eisen:\\
\als{$x < y$}\\
\uitwerking{
Moet gelden dat: $\Rightarrow \alpha(x) < \alpha(y)$\\
Zij $x,y\in K$ met $x < y$\\
Dan $\exists r \in \Q$ zdd $x < r < s < y$\\
Dus $r \in C_y$ maar $r \not\in C_x$\\
$C_x \subseteq C_y$, dus $C_x \subset C_y$\\
dus $C_x < C_y$ (orde van $\R$)\\
dus $\alpha(x) = sup(C_x)\leq r < s \leq sup(C_y) = \alpha(y)$\\
dus $\alpha(x)$ is injectief en orde behoudend!
}\\
Omgekeerd werkt dit ook met $\beta: L \to K$\\
is een inverse afbeelding van $\alpha$\\
dus $\beta(\alpha(x))=x$ en $\alpha(\beta(y))=y$\\
dus $\alpha \wedge \beta$ zijn bijecties!\\
$\beta(\alpha(x))=sup(\{r\in \Q |r < sup(\{s\in \Q | s<x\})\})=x$
}

\feit{$\frac{1}{n}$ is een nulrij}
\bewijs{
	gegeven een $\epsilon>0$ dan:\\
	$\epsilon>0\implies \frac{1}{\epsilon}>0\implies \exists_{n\in\mathbb{R}}n>\frac{1}{\epsilon}\implies\frac{1}{n}<\epsilon$.\\
	We weten dat deze $n$ bestaat omdat $\mathbb{R}$ archimedisch is.\\
	voor $N\geq n$: $N\geq n\implies\frac{1}{N}\leq\frac{1}{n}<\epsilon\implies\frac{1}{N}<\epsilon$
}

\feit{als een rij een limiet heeft dan is deze uniek}
\bewijs{
	stel $k,l$ zijn limieten: $\{a_n\}\to k \land \{a_n\}\to l$ dan:\\
	$\forall_{\epsilon>0}\exists_{n\in\mathbb{R}}\forall_{N>n}(\left|a_N-k\right|<\epsilon\land\left|a_N-l\right|<\epsilon)$\\
	neem dan $\epsilon=\frac{\left|k-l\right|}{3}$:\\
	$\left|a_N-k\right|<\frac{\left|k-l\right|}{3}\implies a_N<k+\frac{\left|k-l\right|}{3}$\\
	$\implies \frac{\left|k-l\right|}{3}>\left|a_N-l\right|>\frac{2\left|k-l\right|}{3}\implies k-l=0\implies k=l$\\
	Dus als er een limiet bestaat, is deze uniek
}


\stelling{Zij $0\leq y \in \mathbb{R}$ en $1 > n \in \N$, dan is er een unieke $0 \leq s \in \mathbb{R}$ zdd. $s^n = y$}
\bewijs{
	Eenheid van de oplossing:\\
	\uitwerking{
		$(a^n-b^n) = (a-b)(a^{n-1} + a^{n-2}b + ... + b^{n-1})$\\
		Zij $a,b \geq 0 \wedge a^n = b^n \Rightarrow a = b$\\ (want tweede deel $\geq 0$ dus $(a-b)=0$
	}\\
	Existentie:\\
	\uitwerking{
	Def $B=\{x\in \mathbb{R}|x\geq0, x^n\leq\})$\\
	$0\in B$, dus $B \neq \emptyset$
	}\\
	Bewering: $B$ is naar boven begrensd\\
	\uitwerking{
		\als $y \leq 1 \wedge x \in B$ dan geldt $x^n \leq y \leq 1 \to x \leq 1$\\	
		Dus 1 is een bovengrens voor verzameling B\\
		\als $y \geq 1 \to y^n \geq y \wedge x\in B$\\
		Dan geldt $0 \geq x^n \geq y \geq y^n \Rightarrow x \geq y$\\
		Dus B is naar boven begrensd door y
	}\\
	Omdat B naar boven begrensd is en $\mathbb{R}$ volledig is,\\
	geldt $0 \geq sup(B) \in \mathbb{R}$\\
	Def $s := sup(B)$\\
	\lemma{\\
	Als $0 < \epsilon < 1 \wedge n\in \mathbb{N}$\\
	Dan $1-n\epsilon \leq (1 - \epsilon)^n<(1+\epsilon)^n\leq 1 + (2^n-1)\epsilon$	
	}
	\uitwerking{
		Inductie over n:\\
		Inductiebegin: $n-\epsilon \leq (1 - \epsilon) < 1 + \epsilon \leq 1 + \epsilon$\\
		Inductiehypothese: stel (*) is waar voor $n \in \mathbb{N}$\\
		$(1-\epsilon)^{n+1}=(1-\epsilon)(1-\epsilon)^n\geq(1-\epsilon)(1-n\epsilon)=..$\\
		$..=1-(n+1)\epsilon+n\epsilon^2 > 1 - 	(n+1)\epsilon$\\
		en ook:\\
		$(1+\epsilon^{n+1}=(1+\epsilon)(1+\epsilon)^n) \leq ..$\\
		$..(1+\epsilon)(1+(2^n -1)\epsilon)$\\
		$..=1+\epsilon2^n+(2^n -1)\epsilon^2 < 1 + \epsilon2^n + (2_n-1)\epsilon=..$\\
		$..=1+\epsilon(2^{n+1}-1)$
	}\\
	Stel $s^n < y$\\
	Gezien $y - s^n > 0$ kunnen we $\eta \in \mathbb{R}$ kiezen\\
	zdd. $0 < \eta < \frac{y-s^n}{2^n-1}s^n$\\
	Dan met gebruik lemma:\\
	$((1+\eta)s)^n \leq s^n +(2^n -1)\eta s^n < s^n + (y - s^n) = y$\\
	Dus $(1 + \eta)s \in B$ Tegenspraak!\\
	Want $\eta > 1$ en $s = sup(B)$\\
	Andere helft van het bewijs ($s^n > y$):\\
	Werkt met het andere deel van de ongelijkheid\\
	uit het lemma. Samen zeggen ze dat $s^n = y$
}

\end{multicols}